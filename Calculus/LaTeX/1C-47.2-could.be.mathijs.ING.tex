\documentclass[a4paper]{article}
\usepackage{amsfonts}
\usepackage{amsmath}
\usepackage{amssymb}
\usepackage{graphicx}
\newcommand{\cis}{\operatorname{cis}}

\begin{document}
  
\noindent \large Auteur: CouldBeMathijs \\
\noindent \large Studierichting: Informatica\\
\noindent \large Oefeningenreeks 1C nr. 47.2\\

\medskip

\normalsize

\begin{eqnarray*}
z^5 & = & -1 \\
    & = & 1 \cdot \cis \pi\\
    & = & \cis \left( \dfrac{\pi + 2\pi k}{5} \right), \forall k \in \mathbb{N} \cap [0, 4]. \\
\end{eqnarray*}
Dus de geldige wortels zijn: \\

\begin{eqnarray*}
\smallskip
z_0 & = & \cis \dfrac{\pi}{5} \\
\smallskip
z_1 & = & \cis \dfrac{3\pi}{5} \\
\smallskip
z_2 & = & \cis \dfrac{5\pi}{5} = -1 \\
\smallskip
z_3 & = & \cis \dfrac{7\pi}{5} \\
\smallskip
z_4 & = & \cis \dfrac{9\pi}{5}
\end{eqnarray*}


\begin{thebibliography}{999}
%\bibitem{a} Referentie
\end{thebibliography}
\end{document} 
