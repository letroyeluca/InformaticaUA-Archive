\documentclass[a4paper]{article}
\usepackage{amsfonts}
\usepackage{amsmath}
\usepackage{amssymb}
\usepackage{graphicx}
\newcommand{\Bgtan}{\operatorname{Bgtan}}

\begin{document}
  
\noindent \large Auteur: CouldBeMathijs \\
\noindent \large Studierichting: Informatica\\
\noindent \large Oefeningenreeks 2B nr. 6.6\\

\medskip

\normalsize

$2 \Bgtan \dfrac{1}{2} - \Bgtan \dfrac{1}{7} = \dfrac{\pi}{4}$ want\\

$2 \Bgtan \dfrac{1}{2} - \Bgtan \dfrac{1}{7}$\\

$=\left( \Bgtan \dfrac{1}{2} + \Bgtan \dfrac{1}{2} \right) -  \Bgtan \dfrac{1}{7}$\\

$=\Bgtan \dfrac{1}{1-\dfrac{1}{4}} - \Bgtan \dfrac{1}{7}$\\

$=\Bgtan \dfrac{4}{3} - \Bgtan \dfrac{1}{7}$\\

$=\Bgtan \dfrac{\dfrac{4}{3} - \dfrac{1}{7}}{1+ \dfrac{4}{21}} = \Bgtan 1 = \dfrac{\pi}{4}$


\begin{thebibliography}{999}
%\bibitem{a} Referentie
\end{thebibliography}
\end{document} 
