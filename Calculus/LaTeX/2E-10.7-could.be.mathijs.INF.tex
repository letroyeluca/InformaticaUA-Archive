\documentclass[a4paper]{article}
\usepackage{amsfonts}
\usepackage{amsmath}
\usepackage{amssymb}
\usepackage{graphicx}     

\begin{document}
  
\noindent \large Auteur: CouldBeMathijs \\
\noindent \large Studierichting: Informatica\\
\noindent \large Oefeningenreeks 2E nr. 10.7\\

\medskip

\normalsize

\begin{eqnarray*}
    2^{2x} + 7 &=& 3 \cdot 2^{1-x} + 2^{2x+1} \\
    \text{\textbf{Neem }} t & = &2^x\\
    t^2 + 7 & = & \dfrac{3 \cdot 2}{t^2} + 2t^2\\
    u^3 + 7u & = & 6 + 2u^3\\
    -u^3 + 7u -6 & = & 0
\end{eqnarray*}

\textbf{Horner:\\}

$\begin{array}{r|r r r r}
    & -1 & 0 & 7 & -6\\
   1&    &-1 &-1 &  6\\ \hline
    & -1 & -1& 6 & 0
\end{array}$ \\

Nieuwe functie: $(t^2-t+6)(t-1) = 0$\\

\textbf{Som- en verschilformules:}\\

$s= -\dfrac{b}{a} = -1$ en $p = \dfrac{c}{a} = -6$\\

$\Rightarrow t_1 = 1$, $t_2 = 2$ en $t_3 = -3$\\

\textbf{t opnieuw vervangen:}\\

$2^x = 1 \Rightarrow x = 0$\\

$2^x = 2 \Rightarrow x = 1$\\

$2^x = -3$ \textit{heeft geen reële oplossingen}




\begin{thebibliography}{999}
%\bibitem{a} Referentie
\end{thebibliography}
\end{document} 
