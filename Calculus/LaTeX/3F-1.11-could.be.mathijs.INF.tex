\documentclass[a4paper]{article}
\usepackage{amsfonts}
\usepackage{amsmath}
\usepackage{amssymb}
\usepackage{graphicx}
\newcommand{\llim}{\lim\limits}

\begin{document}
  
\noindent \large Auteur: CouldBeMathijs \\
\noindent \large Studierichting: Informatica\\
\noindent \large Oefeningenreeks 3F nr. 1.11\\

\medskip

\normalsize

\textbf{Verticale Asymptoot:}\\

$ \llim_{x \rightarrow 7} \left( x + 1 + \dfrac{3}{x-7}\right) = \llim_{x \rightarrow 7} \left( \dfrac{x^2-6x-4}{x-7} \right) = \infty \Rightarrow $ Dé limiet bestaat niet\\

$\llim_{x \rightarrow 7+} = \dfrac{3}{0^+} = +\infty$\\

$\llim_{x \rightarrow 7-} = \dfrac{3}{0^-} = -\infty$\\

Verticale Asymptoot op $x=7$\\

$\begin{array}{r|c c c c}
x & & 7 & &\\ \hline
f(x) & + & | & - &
\end{array}$ \\



\textbf{Horizontale Asymptoot:}\\

$ \llim_{x \rightarrow \pm \infty}\left( x + 1 + \dfrac{3}{x-7}\right) = \pm \infty \Rightarrow $ Geen Horizontale Asymptoot\\

\textbf{Schuine Asymptoot:}\\

$a = \llim_{x \rightarrow \infty}\left(\dfrac{f(x)}{x}\right) = \llim_{x \rightarrow \infty}\left(1+\dfrac{1}{x}+\dfrac{3}{(x-7)(x)}\right) = 1 $\\

$b = \llim_{x \rightarrow \infty}(f(x)-ax) = \llim_{x \rightarrow \infty}\left(1+\dfrac{3}{x-7}\right) = 1$\\

$\Rightarrow$ Schuine Asymptoot: $y=x+1$\\

$\begin{array}{r|c c c}
x & & 7 & \\ \hline
f(x) - (x+1) & + & | & - 
\end{array}$ \\

Conclusie: als $x \rightarrow -\infty$ nadert de functie de schuine asymptoot langs onder, 

en als $x \rightarrow +\infty$ nadert ze hem langs boven.


\begin{thebibliography}{999}
%\bibitem{a} Referentie
\end{thebibliography}
\end{document} 
