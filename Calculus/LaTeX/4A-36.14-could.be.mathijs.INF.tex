\documentclass[a4paper]{article}
\usepackage{amsfonts}
\usepackage{amsmath}
\usepackage{amssymb}
\usepackage{graphicx}     
\newcommand{\intt}{\displaystyle\int}
\newcommand{\Bgtan}{\operatorname{Bgtan}}

\begin{document}
  
\noindent \large Auteur: CouldBeMathijs \\
\noindent \large Studierichting: Informatica\\
\noindent \large Oefeningenreeks 4A nr. 36.14\\

\medskip

\normalsize

$\intt \dfrac{4dx}{2x^2+7} = \dfrac{4}{7} \ \intt \dfrac{dx}{\left(\sqrt{\dfrac{2}{7}} \ x\right)^2+1}$\\

Neem $u = \sqrt{\dfrac{2}{7}} \ x$\\

$du = \sqrt{\dfrac{2}{7}} \ dx \Rightarrow dx = \sqrt{\dfrac{7}{2}} du$\\

Dan $\intt \dfrac{4dx}{2x^2+7} = \dfrac{4 \sqrt{7}}{7 \sqrt{2}} \intt \dfrac{du}{u^2+1}$\\

$= \dfrac{2 \sqrt{14}}{7} \Bgtan \left( \sqrt{\dfrac{2}{7}}x \right) + c$ met $c \in \mathbb{R}$

\begin{thebibliography}{999}
%\bibitem{a} Referentie
\end{thebibliography}
\end{document} 
