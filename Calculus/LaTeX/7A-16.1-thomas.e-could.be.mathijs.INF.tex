\documentclass[a4paper]{article}
\usepackage{amsfonts}
\usepackage{amsmath}
\usepackage{amssymb}
\usepackage{graphicx}
\newcommand{\llim}{\lim\limits}
% Partiële afgeleide macro 
\newcommand{\pa}[1]{\dfrac{\partial f}{\partial #1}}

\begin{document}
  
\noindent \large Auteur: Thomas E. - CouldBeMathijs \\
\noindent \large Studierichting: Informatica\\
\noindent \large Oefeningenreeks 7A nr. 16.1\\

\medskip

\normalsize

$f: \mathbb{R}^2 \to \mathbb{R}:
f(x,y) = \left\{
\begin{array}{ll}
	\dfrac{xy}{\sqrt{x^2 + y^2}} & \text{als } (x,y) \neq (0,0) \\
	0 & \text{als } (x,y) = (0,0)
\end{array}
\right.$\\

\textbf{Partiële integratie op $(0,0)$?}\\

\begin{eqnarray*}
	\pa{x} (0,0) & = & \llim_{h \to 0} \dfrac{f(0+h,0)-f(0,0)}{h}\\
		& = & \llim_{h \to 0} \dfrac{\dfrac{0}{\sqrt{h^2+0}}-0}{h}\\
		& = & \llim_{h \to 0} \dfrac{0}{h^2} = 0\\
	\pa{y} (0,0) & = & \llim_{h \to 0} \dfrac{f(0+h,0)-f(0,0)}{h}\\
		& = & \llim_{h \to 0} \dfrac{\dfrac{0}{\sqrt{h^2+0}}-0}{h}\\
		& = & \llim_{h \to 0} \dfrac{0}{h^2} = 0\\
\end{eqnarray*}

\textbf{Continuïteit op $(0,0)$?}\\

$f(0,0) = 0$\\



Neem $ x = r \cos\theta, y = r \sin \theta$\\

$\llim_{(x,y) \to (0,0)} \dfrac{xy}{\sqrt{x^2+y^2}} = \llim_{r \to 0} \dfrac{r^2 \cos\theta\sin\theta}{\sqrt{r^2(\cos^2\theta+\sin^2\theta)}} = \llim_{r \to 0} \dfrac{r^2 \cos\theta \sin\theta}{|r|}$\\

$\sqrt{x^2+y^2} \ge 0$ want is een afstand $ \Rightarrow \sqrt{r^2} = |r| = r$\\

$\llim_{r \to 0} (r \cos\theta \sin\theta) = 0$\\



\newpage

\textbf{Afleidbaarheid op $(0,0)$?}\\

Voor elke $h = (h_1, h_2)$\\

\begin{eqnarray*}
	Df((0,0), h) &=& \llim_{\lambda \to 0} \dfrac{f(0+\lambda h_1, 0+ \lambda h_2)-f(0,0)}{\lambda}\\
	&=& \llim_{\lambda \to 0} \dfrac{\dfrac{\lambda h_1\lambda h_2}{\sqrt{(\lambda h_1)^2 + (\lambda h_2)^2}}-0}{\lambda}\\
	&=& \llim_{\lambda \to 0} \dfrac{\lambda^2 h_1 h_2}{\lambda \sqrt{\lambda^2 h_1^2 + \lambda^2 h_2^2}}\\
	&=& \llim_{\lambda \to 0} \dfrac{\lambda^2 h_1 h_2}{|\lambda|\lambda \sqrt{h_1^2 + h_2^2}}\\
\end{eqnarray*}

De linkerlimiet en rechterlimiet zijn niet gelijk $\Rightarrow$ f is niet afleidbaar\\

\textbf{Differentieerbaarheid op $(0,0)$?}\\

$\Rightarrow f$ is partieel afleidbaar, continu, maar niet afleidbaar in $(0,0)$\\

$ \Rightarrow f$ \textbf{is niet differentieerbaar op $(0,0)$.}

\begin{thebibliography}{999}
%\bibitem{a} Referentie
\end{thebibliography}
\end{document} 
